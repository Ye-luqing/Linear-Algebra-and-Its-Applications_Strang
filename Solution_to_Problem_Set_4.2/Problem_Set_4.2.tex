%=======================02-713 LaTeX template, following the 15-210 template==================
%
% You don't need to use LaTeX or this template, but you must turn your homework in as
% a typeset PDF somehow.
%
% How to use:
%    1. Update your information in section "A" below
%    2. Write your answers in section "B" below. Precede answers for all 
%       parts of a question with the command "\question{n}{desc}" where n is
%       the question number and "desc" is a short, one-line description of 
%       the problem. There is no need to restate the problem.
%    3. If a question has multiple parts, precede the answer to part x with the
%       command "\part{x}".
%    4. If a problem asks you to design an algorithm, use the commands
%       \algorithm, \correctness, \runtime to precede your discussion of the 
%       description of the algorithm, its correctness, and its running time, respectively.
%    5. You can include graphics by using the command \includegraphics{FILENAME}
%
\documentclass[11pt]{article}
\usepackage{amsmath,amssymb,amsthm}
\usepackage{graphicx}
\usepackage[margin=1in]{geometry}
\usepackage{fancyhdr}
\setlength{\parindent}{0pt}
\setlength{\parskip}{5pt plus 1pt}
\setlength{\headheight}{13.6pt}
\newcommand\question[2]{\vspace{.25in}\hrule\textbf{#1: #2}\vspace{.5em}\hrule\vspace{.10in}}
\renewcommand\part[1]{\vspace{.10in}\textbf{(#1)}}
\newcommand\solution{\vspace{.10in}\textbf{Solution: }}
\newcommand\correctness{\vspace{.10in}\textbf{Correctness: }}
\newcommand\runtime{\vspace{.10in}\textbf{Running time: }}
\pagestyle{fancyplain}
\lhead{\textbf{\TITLE}}
\rhead{\today}
\begin{document}\raggedright
\newcommand\TITLE{Linear Algebra and Its Applications:Solutions to Problem Set 4.2} 


\question{1}{If a $4$ by $4$ matrix has $\det A=\frac{1}{2}$,find
  $\det (2A)$,$\det (-A)$,$\det A^2$,and $\det (A^{-1})$.} 
\solution 
\begin{itemize}
\item $\det (2A)=2^4\det A=8$.
\item $\det (-A)=(-1)^4\det A=\frac{1}{2}$.
\item $\det A^2=\det A\cdot \det A=\frac{1}{4}$.
\item $\det (A^{-1})=\frac{1}{\det A}=2$.
\end{itemize}
\question{2}{If a $3$ by $3$ matrix has $\det A=-1$,find $\det
  (\frac{1}{2}A)$,$\det (-A)$,$\det (A^2)$,and $\det (A^{-1})$.}
\solution
\begin{itemize}
\item $\det (\frac{1}{2}A)=(\frac{1}{2})^3\det A=-\frac{1}{8}$.
\item $\det (-A)=(-1)^3\det A=1$.
\item $\det (A^2)=\det A\cdot \det A=1$.
\item $\det (A^{-1})=\frac{1}{\det A}=-1$.
\end{itemize}
\question{3}{\textit{Row exchange}:Add row $1$ to row $2$,then
  subtract row $2$ from row $1$.Then add row $1$ to row $2$ and
  multiply row $1$ by $-1$ to reach $B$.Which rules show the
  following?
$$
\det B=
\begin{vmatrix}
  c&d\\
  a&b
\end{vmatrix}\mbox{equals}-\det A=-
\begin{vmatrix}
  a&b\\
  c&d
\end{vmatrix}.
$$
Those rules could replace Rule $2$ in the definition of the determinant.
}
\solution Let $A=
\begin{pmatrix}
  a&b\\
  c&d
\end{pmatrix}.
$Then,the operation flow mentioned in the exercise can be represented
as below:
$$
A=\begin{pmatrix}
  a&b\\
  c&d
\end{pmatrix}\to
\begin{pmatrix}
  a&b\\
 a+c&b+d
\end{pmatrix}\to
\begin{pmatrix}
  -c&-d\\
  a+c&b+d
\end{pmatrix}\to
\begin{pmatrix}
  -c&-d\\
  a&b
\end{pmatrix}\to
\begin{pmatrix}
  c&d\\
  a&b
\end{pmatrix}=B.
$$
Rule 5:Subtracting a multiple of one row from another row leaves the
same determinant.Rule 5 can replace Rule 2.
\question{4}{By applying row operations to produce an upper triangular
$U$,compute
$$
\det
\begin{bmatrix}
  1&2&-2&0\\
  2&3&-4&1\\
  -1&-2&0&2\\
  0&2&5&3
\end{bmatrix}\mbox{and} ~\det
\begin{bmatrix}
  2&-1&0&0\\
 -1&2&-1&0\\
0&-1&2&-1\\
0&0&-1&-2
\end{bmatrix}.
$$
Exchange rows $3$ and $4$ of the second matrix and recompute the
pivots and determinant.
}
\solution
\begin{itemize}
\item Let $A=\begin{bmatrix}
  1&2&-2&0\\
  2&3&-4&1\\
  -1&-2&0&2\\
  0&2&5&3
\end{bmatrix}$.Then we perform row operations to matrix $A$:
\begin{align*}
A=\begin{bmatrix}
  1&2&-2&0\\
  2&3&-4&1\\
  -1&-2&0&2\\
  0&2&5&3
\end{bmatrix}&\xrightarrow{row 2+row 1\times (-2)}
               \begin{bmatrix}
                 1&2&-2&0\\
                0&-1&0&1\\
               -1&-2&0&2\\
              0&2&5&3
               \end{bmatrix}\xrightarrow{row 3+row 1}
                     \begin{bmatrix}
                       1&2&-2&0\\
                      0&-1&0&1\\
                     0&0&-2&2\\
                    0&2&5&3
                     \end{bmatrix}\\&\xrightarrow{row 4+row 2\times 2}
                           \begin{bmatrix}
                             1&2&-2&0\\
0&-1&0&1\\
0&0&-2&2\\
0&0&5&5
                           \end{bmatrix}\xrightarrow{row 4+row 3\times
       \frac{5}{2}}
       \begin{bmatrix}
         1&2&-2&0\\
0&-1&0&1\\
0&0&-2&2\\
0&0&0&10
       \end{bmatrix}=U.
\end{align*}
So $\det A=\det U=20$.
\item Let $B=\begin{bmatrix}
  2&-1&0&0\\
 -1&2&-1&0\\
0&-1&2&-1\\
0&0&-1&-2
\end{bmatrix}.$Then we perform row operations to matrix $B$:
\begin{align*}
  B=\begin{bmatrix}
  2&-1&0&0\\
 -1&2&-1&0\\
0&-1&2&-1\\
0&0&-1&-2
\end{bmatrix}&\xrightarrow{row 2+row 1\times \frac{1}{2}}
               \begin{bmatrix}
                 2&-1&0&0\\
0&\frac{3}{2}&-1&0\\
0&-1&2&-1\\
0&0&-1&-2
               \end{bmatrix}\xrightarrow{row 3+row 2\times \frac{2}{3}}
        \begin{bmatrix}
          2&-1&0&0\\
          0&\frac{3}{2}&-1&0\\
0&0&\frac{4}{3}&-1\\
0&0&-1&-2
        \end{bmatrix}\\&\xrightarrow{row 4+row 3\times \frac{3}{4}}
        \begin{bmatrix}
          2&-1&0&0\\
0&\frac{3}{2}&-1&0\\
0&0&\frac{4}{3}&-1\\
0&0&0&-\frac{11}{4}
        \end{bmatrix}=U.
\end{align*}
So $\det B=\det U=-11$.\\

When row $3$ and $4$ of the second matrix are changed,the matrix $B$
turned into the matrix $B'=
\begin{bmatrix}
  2&-1&0&0\\
  -1&2&-1&0\\
  0&0&-1&-2\\
 0&-1&2&-1
\end{bmatrix}
$.Perform row operations to the matrix $B'$,we get matrix $U'=
\begin{bmatrix}
  2&-1&0&0\\
0&\frac{3}{2}&-1&0\\
0&0&-1&-2\\
0&0&0&-\frac{11}{3}
\end{bmatrix}
$.So $\det B'=\det U'=11$.
\end{itemize}
\question{5}{Count row exchanges to find these determinants:
$$
\det
\begin{bmatrix}
  0&0&0&1\\
  0&0&1&0\\
  0&1&0&0\\
  1&0&0&0
\end{bmatrix}=\pm 1~\mbox{and}~\det
\begin{bmatrix}
  0&1&0&0\\
0&0&1&0\\
0&0&0&1\\
1&0&0&0
\end{bmatrix}=-1.
$$
}
\solution Exchange row 1 and 4,then row 2 and 3 of the first matrix,we get an identity matrix.So
the determinant of the first matrix is $1$.\\

Exchange row 1 and 4,then row 3 and 4,then row 2 and row 3
of the second matrix,we get an identity matrix.So the determinant of
the second matrix is $-1$.
\question{6}{For each $n$,how many exchanges will put (row $n$,row
  $n-1$,$\cdots$,row $1$) into the normal order (row $1$,$\cdots$,row
  $n-1$,row $n$)?Find $\det P$ for the $n$ by $n$ permutation with
  $1$s on the reverse diagonal.Problem $5$ had $n=4$.}
\solution $(n-1)+(n-2)+\cdots+1=\frac{n(n-1)}{2}$.Putting (row $n$,row
  $n-1$,$\cdots$,row $1$) into the normal order (row $1$,$\cdots$,row
  $n-1$,row $n$) requires $\frac{n(n-1)}{2}$ row exchanges.$\det
  P=(-1)^{\frac{n(n-1)}{2}}$.
\question{7}{Find the determinants of 
\begin{itemize}
\item a rank one matrix
$$
A=
\begin{bmatrix}
  1\\
4\\
2
\end{bmatrix}
\begin{bmatrix}
  2&-1&2
\end{bmatrix}.
$$
\item the upper triangular matrix
$$
U=
\begin{bmatrix}
  4&4&8&8\\
0&1&2&2\\
0&0&2&6\\
0&0&0&2
\end{bmatrix}.
$$
\item the lower triangular matrix $U^T$.
\item the inverse matrix $U^{-1}$.
\item the "reverse-triangular" matrix that results from row exchanges,
$$
M=
\begin{bmatrix}
  0&0&0&2\\
0&0&2&6\\
0&1&2&2\\
4&4&8&8
\end{bmatrix}.
$$
\end{itemize}
}
\solution
\begin{itemize}
\item $\det A=0$.
\item $\det U=16$.
\item $\det U^T=16$.
\item $\det U^{-1}=\frac{1}{16}$.
\item $\det M=(-1)^2\det U=\det U=16$.
\end{itemize}
\question{8}{Show  how rule $6$($\det=0$ if a row is zero) comes directly from rules $2$ and
  $3$.}
\solution $
\begin{vmatrix}
  0&0\\
  c&d
\end{vmatrix}=
\begin{vmatrix}
  c-c&d-d\\
  c&d
\end{vmatrix}=
\begin{vmatrix}
  c&d\\
  c&d
\end{vmatrix}+
\begin{vmatrix}
  -c&-d\\
  c&d
\end{vmatrix}=
\begin{vmatrix}
  c&d\\
  c&d
\end{vmatrix}-
\begin{vmatrix}
  c&d\\
c&d
\end{vmatrix}=0.
$
\question{9}{Suppose you do two row operations at once,going from 
$$
\begin{bmatrix}
  a&b\\
  c&d
\end{bmatrix}~\mbox{to}~
\begin{bmatrix}
  a-mc&b-md\\
  c-la&d-lb
\end{bmatrix}.
$$
Find the determinant of the new matrix,by rule 3 or by direct calculation.
}
\solution \begin{align*}
\begin{vmatrix}
  a-mc&b-md\\
  c-la&d-lb
\end{vmatrix}&=
\begin{vmatrix}
  a&b\\
 c-la&d-lb
\end{vmatrix}-
\begin{vmatrix}
  mc&md\\
  c-la&d-lb
\end{vmatrix}\\&=
\begin{vmatrix}
  c-la&d-lb\\
  mc&md
\end{vmatrix}-
\begin{vmatrix}
  c-la&d-lb\\
 a&b
\end{vmatrix}\\&=\left(
                 \begin{vmatrix}
                   c&d\\
mc&md
                 \end{vmatrix}-
    \begin{vmatrix}
      la&lb\\
mc&md
    \end{vmatrix}
\right)-\left(
    \begin{vmatrix}
      c&d\\
     a&b
    \end{vmatrix}-
        \begin{vmatrix}
          la&lb\\
           a&b
        \end{vmatrix}
\right)
\\&=-
    \begin{vmatrix}
      la&lb\\
      mc&md
    \end{vmatrix}-
          \begin{vmatrix}
            c&d\\
           a&b
          \end{vmatrix}
\\&=-ml\left(
    \begin{vmatrix}
      a&b\\
     c&d
    \end{vmatrix}\right)+
        \begin{vmatrix}
          a&b\\
c&d
        \end{vmatrix}
\\&=(1-ml)(ad-bc).
\end{align*}
\question{10}{If $Q$ is an orthogonal matrix,so that $Q^TQ=I$,prove
  that $\det Q$ equals $+1$ or $-1$.What kind of box is formed from
  the rows(or columns)of $Q$?}
\solution $\det Q^TQ=\det Q\det Q=\det I=1$,so $\det Q=\pm 1$.The rows
or columns of $Q$ form a  $n$ dimensional unit cube,in which $n$ is
the rank of $Q$.
\question{11}{Prove again that $\det Q=1$ or $-1$ using only the
  Product rule.If $|\det Q|>1$ then $\det Q^n$ blows up.How do you
  know this can't happen to $Q^n$?}
\solution Because $Q^n$ is still an orthogonal matrix.Its determinant
can't blow up because the rows of $Q^n$ form an $n$ dimensional unit
cube.
\question{12}{Use row operations to verify that the $3$ by $3$
  "Vandermonde determinant" is
$$
\det
\begin{bmatrix}
  1&a&a^2\\
  1&b&b^2\\
  1&c&c^2
\end{bmatrix}=(b-a)(c-a)(c-b).
$$
}
\solution When either two of $a,b,c$ are equal,both the determinant
and $(b-a)(c-a)(c-b)$ are zero.So in this case the identity holds.When
neither two of $a,b,c$ are equal,
\begin{align*}
  \det
  \begin{bmatrix}
    1&a&a^2\\
    1&b&b^2\\
    1&c&c^2
  \end{bmatrix}&=
                 \det\begin{bmatrix}
                   1&a&a^2\\
                   0&b-a&b^2-a^2\\
                   0&c-a&c^2-a^2
                 \end{bmatrix}=
                         \det\begin{bmatrix}
                            1&a&a^2\\
0&b-a&b^2-a^2\\
0&0&(c^2-a^2)-\frac{b^2-a^2}{b-a}(c-a)
                          \end{bmatrix}
\\&=\det
    \begin{bmatrix}
      1&a&a^2\\
      0&b-a&b^2-a^2\\
      0&0&(c-a)(c-b)
    \end{bmatrix}.
\end{align*}
So $\det
\begin{bmatrix}
  1&a&a^2\\
1&b&b^2\\
1&c&c^2
\end{bmatrix}=(b-a)(c-a)(c-b).
$
\question{13}{
  \begin{itemize}
  \item A skew-symmetric matrix satisfies $K^T=-K$,as in 
$$
K=
\begin{bmatrix}
  0&a&b\\
-a&0&c\\
-b&-c&0
\end{bmatrix}.
$$
In the $3$ by $3$ case,why is $\det (-K)=(-1)^3\det K$?On the other
hand $\det K^T=\det K$(always).Deduce that the determinant must be
zero.
\item Write down a $4$ by $4$ skew-symmetric matrix with $\det K$ not zero.
  \end{itemize}
}
\solution
\begin{itemize}
\item $\det (-K)=(-1)^3\det K=-\det K$.$\det (-K)=\det K^T=\det K$.So
  $\det K=-\det K$,so $\det K=0$.
\item $
  \det\begin{bmatrix}
    0&0&0&1\\
    0&0&1&0\\
0&-1&0&0\\
-1&0&0&0
  \end{bmatrix}\neq 0.
$
\end{itemize}
\question{14}{True or false,with reason if true and counterexample if
  false:
  \begin{itemize}
  \item If $A$ and $B$ are identical except that $b_{11}=2a_{11}$,then
    $\det B=2\det A$.
\item The determinant is the product of the pivots.
\item If $A$ is invertible and $B$ is singular,then $A+B$ is
  invertible.
\item If $A$ is invertible and $B$ is singular,then $AB$ is singular.
\item The determinant of $AB-BA$ is zero.
  \end{itemize}
}
\solution
\begin{itemize}
\item False.Counterexample:$
\det  \begin{pmatrix}
    1&1\\
1&1
  \end{pmatrix}=0,
\det  \begin{pmatrix}
    2&1\\
1&1
  \end{pmatrix}=1.
$
\item True.
\item False.Counterexample:$A=
  \begin{bmatrix}
    1&0\\
    0&1
  \end{bmatrix},b=
  \begin{bmatrix}
    -1&0\\
     0&0
  \end{bmatrix}.
$
\item True.
\item False.$$ A=
  \begin{bmatrix}
    1&2\\
    -1&2
  \end{bmatrix},B=
  \begin{bmatrix}
    1&3\\
0&-1
  \end{bmatrix},
$$then $\det (AB-BA)=\det
\begin{bmatrix}
  3&-7\\
-2&-3
\end{bmatrix}\neq 0.
$
\end{itemize}
\question{15}{If every row of $A$ adds to zero,prove that $\det
  A=0$.If every row adds to $1$,prove that $\det (A-I)=0$.Show by
  example that this does not imply $\det A=I$.}
\solution When every row of $A$ adds to zero,then $A$ is singular,so
$\det A=0$.\\
When every row of $A$ adds to one,then every row of $A-I$ adds to
zero,so $\det (A-I)=0$.\\

$\det (A-I)=0$ does not imply that $\det A=1$.For example,$A=
\begin{bmatrix}
  1&2\\
  0&2
\end{bmatrix}.
$
\question{16}{Find these $4$ by $4$ determinants by Gaussian
  elimination:}
$$
\det
\begin{bmatrix}
  11&12&13&14\\
  21&22&23&24\\
  31&32&33&34\\
  41&42&43&44
\end{bmatrix}~\mbox{and}~\det
\begin{bmatrix}
  1&t&t^2&t^3\\
  t&1&t&t^2\\
  t^2&t&1&t\\
  t^3&t^2&t&1
\end{bmatrix}.
$$
\solution 
$$
  \det
  \begin{bmatrix}
    11&12&13&14\\
    21&22&23&24\\
    31&32&33&34\\
    41&42&43&44
  \end{bmatrix}=\det
                 \begin{bmatrix}
                   11&12&13&14\\
-1&-2&-3&-4\\
-2&-4&-6&-8\\
-3&-6&-9&-12
                 \end{bmatrix}=0.
$$
\begin{align*}
\det
\begin{bmatrix}
  1&t&t^2&t^3\\
  t&1&t&t^2\\
t^2&t&1&t\\
t^3&t^2&t&1
\end{bmatrix}&=\det
\begin{bmatrix}
  1&t&t^2&t^3\\
  0&1-t^2&t-t^3&t^2-t^4\\
  0&t-t^3&1-t^4&t-t^5\\
  0&t^2-t^4&t-t^5&1-t^6
\end{bmatrix}
\\&=\det
    \begin{bmatrix}
      1&t&t^2&t^3\\
      0&1-t^2&1-t^3&t^2-t^4\\
      0&0&1-t^2&t-t^{3}\\
      0&0&t-t^{3}&1-t^4
    \end{bmatrix}
\\&=\det
    \begin{bmatrix}
      1&t&t^2&t^3\\
0&1-t^2&1-t^3&t^2-t^4\\
0&0&1-t^2&t-t^3\\
0&0&0&1-t^2
    \end{bmatrix}
\\&=(1-t^2)^3.
\end{align*}
\question{17}{Find the determinant of 
$$
A=
\begin{bmatrix}
  4&2\\
1&3
\end{bmatrix},A^{-1}=\frac{1}{10}
\begin{bmatrix}
  3&-2\\
-1&4
\end{bmatrix},A-\lambda I=
\begin{bmatrix}
  4-\lambda&2\\
1&3-\lambda
\end{bmatrix}.
$$
For which value of $\lambda$ is $A-\lambda I$ a singular matrix?
}
\solution $\det A=\det
\begin{bmatrix}
  4&2\\
1&3
\end{bmatrix}=\det
\begin{bmatrix}
  4&2\\
0&\frac{5}{2}
\end{bmatrix}=10
$.$\det A^{-1}=\frac{1}{\det A}=\frac{1}{10}$.When $\det (A-\lambda
I)=0$,
$$
\det (A-\lambda I)=\det
\begin{bmatrix}
  4-\lambda&2\\
1&3-\lambda
\end{bmatrix}=-\det
\begin{bmatrix}
  1&3-\lambda\\
4-\lambda&2
\end{bmatrix}=-\det
\begin{bmatrix}
  1&3-\lambda\\
 0&2-(3-\lambda)(4-\lambda)
\end{bmatrix}=(3-\lambda)(4-\lambda)-2=0,
$$
so $\lambda=2$or $5$.
\question{18}{Evaluate $\det A$ by reducing the matrix to triangular
  form (rules $5$ and $7$).
$$
A=
\begin{bmatrix}
  1&1&3\\
0&4&6\\
1&5&8
\end{bmatrix},B=
\begin{bmatrix}
  1&1&3\\
0&4&6\\
0&0&1
\end{bmatrix},C=
\begin{bmatrix}
  1&1&3\\
0&4&6\\
1&5&9
\end{bmatrix}.
$$
What are the determinants of $B,C,AB,A^TA,$and $C^T$?
}
\solution 
$$
\det A=\det
\begin{bmatrix}
  1&1&3\\
  0&4&6\\
  0&4&5
\end{bmatrix}=\det
\begin{bmatrix}
  1&1&3\\
0&4&6\\
0&0&1
\end{bmatrix}=4.
$$
$\det B=4$.$\det C^T=\det C=\det
\begin{bmatrix}
  1&1&3\\
  0&4&6\\
  0&4&6
\end{bmatrix}=0.
$
\question{19}{Suppose that $CD=-DC$,and find the flaw in the following
argument:Taking determinant gives $(\det C)(\det D)=-(\det D)(\det
C)$,so either $\det C=0$ or $\det D=0$.Thus $CD=-DC$ is only possible
if $C$ or $D$ is singular.}
\solution $\det -DC= -(\det D)(\det C)$ is generally false.
\question{20}{Do these matrices have determinant $0,1,2,$or $3$?
$$
A=
\begin{bmatrix}
  0&0&1\\
1&0&0\\
0&1&0
\end{bmatrix},B=
\begin{bmatrix}
  0&1&1\\
1&0&1\\
1&1&0
\end{bmatrix},C=
\begin{bmatrix}
  1&1&1\\
1&1&1\\
1&1&1
\end{bmatrix}.
$$
}
\solution $\det A=1$.$\det B=2$.$\det C=0$.
\question{21}{The inverse of a $2$ by $2$ matrix seems to have
  determinant $=1$:
$$
\det A^{-1}=\det \frac{1}{ad-bc}
\begin{bmatrix}
  d&-b\\
-c&a
\end{bmatrix}=\frac{ad-bc}{ad-bc}=1.
$$
What is wrong with this calculation?What is the correct $\det A^{-1}$?
}
\solution $\det A^{-1}=\frac{1}{ad-bc}$.
\question{22}{Reduce $A$ to $U$ and find $\det A=$ product of the
  pivots:
$$
A=
\begin{bmatrix}
  1&1&1\\
1&2&2\\
1&2&3
\end{bmatrix}~\mbox{and}~A=
\begin{bmatrix}
  1&2&3\\
2&2&3\\
3&3&3
\end{bmatrix}.
$$
}
\solution
\begin{itemize}
\item $\det A=\det
  \begin{bmatrix}
    1&1&1\\
   0&1&1\\
0&1&2
  \end{bmatrix}=\det
  \begin{bmatrix}
    1&1&1\\
    0&1&1\\
    0&0&1
  \end{bmatrix}=1.
$
\item $\det A=\det
  \begin{bmatrix}
    1&2&3\\
0&-2&-3\\
0&-3&-6
  \end{bmatrix}=\det
  \begin{bmatrix}
    1&2&3\\
    0&-2&-3\\
    0&0&-\frac{3}{2}
  \end{bmatrix}=3.
$
\end{itemize}
\question{23}{By applying row operations to produce an upper
  triangular $U$,compute
$$
\det
\begin{bmatrix}
  1&2&3&0\\
2&6&6&1\\
-1&0&0&3\\
0&2&0&7
\end{bmatrix}~\mbox{and}~\det
\begin{bmatrix}
  2&1&1&1\\
1&2&1&1\\
1&1&2&1\\
1&1&1&2
\end{bmatrix}.
$$
}
\solution 
$$
\det
\begin{bmatrix}
  1&2&3&0\\
  2&6&6&1\\
  -1&0&0&3\\
  0&2&0&7
\end{bmatrix}=\det
\begin{bmatrix}
  1&2&3&0\\
  0&2&0&1\\
  0&2&3&3\\
  0&2&0&7
\end{bmatrix}=\det
\begin{bmatrix}
  1&2&3&0\\
  0&2&0&1\\
  0&0&3&2\\
  0&0&0&6
\end{bmatrix}=36.
$$
$$
\det
\begin{bmatrix}
  1&t&t^2\\
  t&1&t\\
  t^2&t&1
\end{bmatrix}=\det
\begin{bmatrix}
  1&t&t^2\\
  0&1-t^2&t-t^3\\
  0&t-t^3&1-t^4
\end{bmatrix}=
\begin{bmatrix}
  1&t&t^2\\
  0&1-t^2&t-t^{3}\\
  0&0&1-t^2
\end{bmatrix}=(1-t^2)^2.
$$
\question{25}{Elimination reduces $A$ to $U$.Then $A=LU$:
$$
A=
\begin{bmatrix}
  3&3&4\\
  6&8&7\\
  -3&5&-9
\end{bmatrix}=
\begin{bmatrix}
  1&0&0\\
  2&1&0\\
  -1&4&1
\end{bmatrix}
\begin{bmatrix}
  3&3&4\\
  0&2&-1\\
  0&0&-1
\end{bmatrix}=LU.
$$
Find the determinant of $L,U,A,U^{-1}L^{-1}$,and $U^{-1}L^{-1}A$.
}
\solution $\det L=1$,$\det U=-6$,$\det A=\det L\det U=-6$.$\det
U^{-1}L^{-1}=\det U^{-1}\det L^{-1}=-\frac{1}{6}$.$\det
U^{-1}L^{-1}A=\det I=1$.
\question{26}{If $a_{ij}$ is $i$ times $j$,show that $\det
  A=0$.(Exception when $A=[1]$.)}
\solution Every two rows of $A$ are linearly dependent,so $\det A=0$.
\question{27}{If $a_{ij}$ is $i+j$,show that $\det A=0$.(Exception
  when $n=1$ or $2$).}
\solution Let the second row of $A$ subtract the first row of $A$,we
get a matrix $A'$ in which all the element of the second row are
$2$.Let the third row of $A'$ subtract the second row of $A'$,we get a
matrix $A''$ in which all the element of the third row are $3$.The
second and the third row of $A''$ are linearly dependent,so $\det
A=\det A'=\det A''=0$.
\question{28}{Compute the determinant of these matrices by row
  operations:
$$
A=
\begin{bmatrix}
  0&a&0\\
0&0&b\\
c&0&0
\end{bmatrix},B=
\begin{bmatrix}
  0&a&0&0\\
  0&0&b&0\\
  0&0&0&c\\
  d&0&0&0
\end{bmatrix},~\mbox{and}~C=
\begin{bmatrix}
  a&a&a\\
  a&b&b\\
  a&b&c
\end{bmatrix}.
$$
}
\solution $\det A=abc$.$\det B=-abcd$.
$$
\det C=\det
\begin{bmatrix}
  a&a&a\\
  0&b-a&b-a\\
  0&b-a&c-a
\end{bmatrix}=\det
\begin{bmatrix}
  a&a&a\\
  0&b-a&b-a\\
  0&0&c-b
\end{bmatrix}=a(b-a)(c-b).
$$
\question{29}{What is wrong with this proof that projection matrices
  have $\det P=1$?
$$
P=A(A^TA)^{-1}A^T~\mbox{so}~|P|=|A|\frac{1}{|A^T||A|}|A^T|=1.
$$
}
\solution $A^TA$ is not necessarily nonsingular.
\question{30}{Show that the partial derivatives of $\ln (\det A)$ give
$A^{-1}$:
$$f(a,b,c,d)=\ln (ad-bc)~\mbox{leads to }~
\begin{bmatrix}
  \frac{\partial f}{\partial a}&\frac{\partial f}{\partial c}\\
  \frac{\partial f}{\partial b}&\frac{\partial f}{\partial d}
\end{bmatrix}=A^{-1}.
$$
}
\solution 
$$
\begin{bmatrix}
  \frac{\partial f}{\partial a}&\frac{\partial f}{\partial c}\\
  \frac{\partial f}{\partial b}&\frac{\partial f}{\partial d}
\end{bmatrix}=
\begin{bmatrix}
  \frac{d}{ad-bc}&\frac{-b}{ad-bc}\\
  \frac{-c}{ad-bc}&\frac{a}{ad-bc}
\end{bmatrix}=A^{-1}.
$$
\question{31-33}{Omitted.}
\question{34}{If you know that $\det A=6$,what is the determinant of
  $B$?
$$
\det A=
\begin{vmatrix}
  \mbox{row}1\\
\mbox{row}2\\
\mbox{row}3
\end{vmatrix}=6,\det B=
\begin{vmatrix}
  \mbox{row}1+\mbox{row}2\\
  \mbox{row}2+\mbox{row}3\\
  \mbox{row}3+\mbox{row}1
\end{vmatrix}=
$$
}
\solution $\det B=0$.
\question{35}{Suppose the $4$ by $4$ matrix $M$ has four equal rows
  all containing $a,b,c,d$.We know that $\det (M)=0$.The problem is to
find $\det (I+M)$ by any method:
$$
\det (I+M)=
\begin{vmatrix}
  1+a&b&c&d\\
  a&1+b&c&d\\
  a&b&1+c&d\\
  a&b&c&1+d
\end{vmatrix}.
$$
}
\solution 
\begin{align*}
  \det (I+M)&=\det
  \begin{bmatrix}
    1+a&b&c&d\\
    -1&1&0&0\\
    -1&0&1&0\\
    -1&0&0&1
  \end{bmatrix}=\det
            \begin{bmatrix}
              a&b+1&c&d\\
             -1&1&0&0\\
             -1&0&1&0\\
             -1&0&0&1
            \end{bmatrix}
\\&=-\det
    \begin{bmatrix}
      -1&1&0&0\\
      a&b+1&c&d\\
      -1&0&1&0\\
      -1&0&0&1
    \end{bmatrix}=-\det
              \begin{bmatrix}
                -1&1&0&0\\
                0&a+b+1&c&d\\
                0&-1&1&0\\
0&-1&0&1
              \end{bmatrix}
\\&=\det
    \begin{bmatrix}
      -1&1&0&0\\
      0&-1&1&0\\
      0&a+b+1&c&d\\
     0&-1&0&1
    \end{bmatrix}=\det
             \begin{bmatrix}
               -1&1&0&0\\
               0&-1&1&0\\
               0&0&a+b+c+1&d\\
               0&0&-1&1
             \end{bmatrix}
\\&=-\det
    \begin{bmatrix}
      -1&1&0&0\\
     0&-1&1&0\\
     0&0&-1&1\\
     0&0&a+b+c+1&d
    \end{bmatrix}=-\det
                  \begin{bmatrix}
                    -1&1&0&0\\
0&-1&1&0\\
0&0&-1&1\\
0&0&0&a+b+c+d+1
                  \end{bmatrix}\\
&=a+b+c+d+1.
\end{align*}
\end{document}
